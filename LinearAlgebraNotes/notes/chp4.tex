
\chapter{ 线性空间和线性映射 }
\label{chap:Linear Space and Linear Map}

\section{ 线性空间 }

\begin{definition}[线性空间]
设$\mathbb{F}$是一数域, $\mathbb{F}$是一个非空集合, 设有运算(映射), 称为加法, 从$V \times V \to V$; 再有一运算称为数乘, 从$\mathbb{F} \times V \to V$, 且满足:
\begin{enumerate}[itemindent=1em]
    \item 加法: 满足交换律
    \item 加法: 满足结合律
    \item 加法: 存在单位元
    \item 加法: $\forall \alpha \in V$ 使得$\alpha \beta = \emph{0}$, 记作$\beta = -\alpha$
    \item 乘法: 存在单位元
    \item 对数的加法的分配律
    \item 对向量的加法的分配律
\end{enumerate}
则称$V$是一个关于加法, 数乘的线性空间, 由单点集给出的线性空间称为零空间.
\end{definition}

\begin{example}
    \par
    \begin{enumerate}[itemindent=1em]
        \item $\mathbb{F}[x]$关于多项式的加法, 数乘是$\mathbb{F}$中的一个线性空间.
        \item $\mathbb{F}^n$关于向量的加法, 数乘是$\mathbb{F}$上的一个线性空间.
        \item $\mathbb{M}_(m \times n)$关于矩阵的乘法, 数乘是$\mathbb{F}$上的一个线性空间.
        \item $HOM(\mathbb{F}^n, \mathbb{F}^n)$中映射的加法和数乘.
        \item $C[a, b]$中函数的加法和数乘.
        \item $\forall a \in [a, b], f: a \mapsto 0 $, $f$是一个线性映射
        \item 若$\mathbb{F} \subseteq \mathbb{K}$, 则$\mathbb{K}$是$\mathbb{F}$上的线性空间.
        \item $\mathbb{R}^+$中规定加法为数的乘法, 规定乘法为求幂, 则$\mathbb{R}^+$是$\mathbb{R}$上的线性空间
    \end{enumerate}
\end{example}

\begin{property}
    设$V$是$\mathbb{F}$上的线性空间, 则:
    \begin{enumerate}[itemindent=1em]
        \item 加法满足消去律
        \item $-\alpha = -1 \cdot \alpha$ (对数的加法的分配律)
        \item 若$k\alpha=0$, 则$k=0$或$\alpha=0$ (数域的性质$k \cdot k^{-1}=1$, 反证法)
        \item $\emph{0}$和$-\alpha$的唯一性
    \end{enumerate}
\end{property}

\begin{inference}
    非零向量的不同倍数不相同
\end{inference}

\section{ 维数、基和坐标 }

\begin{definition}[线性组合]
    $\sum^s_{i=1}k_i\alpha{_i} = k_1\alpha{_1} + \cdots + k_s\alpha{_s} \in V$称为$\alpha{_1}, \cdots, \alpha{_s}$的一个线性组合.
\end{definition}

一般称线性空间中的元素为向量, "\emph{0}"为零向量.

\begin{definition}[线性表出]
    设$V$是$\mathbb{F}$上的线性空间, $\beta, \alpha{_1}, \cdots, \alpha{_n} \in V$, 如果$\exists k_1, \cdots, k_n \in \mathbb{F}$使得
    \[\beta = \sum^n_{i=1}k_i\alpha{_i}\]
    则称$\beta$可以由$\alpha{_1}, \cdots, \alpha{_n}$线性表出.
\end{definition}

\begin{example}
    \par
    \begin{enumerate}[itemindent=1em]
        \item \emph{0}可以由任意向量属于$V$线性表出.
        \item $\alpha \in \mathbb{F}^n$可以由基本向量线性表出.
        \item 矩阵可由基本矩阵线性表出
        \item $\forall a \in \mathbb{F}, f(x) \in \mathbb{F}_n[x], f(x)$可以由$a, x-a, (x-a)^2, \cdots, (x-a)^n$线性表出.
        \item $\mathbb{C}$是$\mathbb{C}$上的线性空间, 可以由其线性表出.
        \item $\mathbb{C}$是$\mathbb{R}$上的线性空间, 不可以由其线性表出.
    \end{enumerate}
\end{example}

\begin{theorem}
    $\forall \beta, \alpha{_1}, \cdots, \alpha{_n}$只有下面三种情况之一:
    \begin{enumerate}[itemindent=1em]
        \item $\beta$不可以由其线性表出.
        \item $\beta$可以由其线性表出, 且方式唯一.
        \item $\beta$可以由其线性表出, 且方式不唯一.
    \end{enumerate}
\end{theorem}

\begin{theorem}
    如果$\beta$可以由$\alpha{_1}, \cdots, \alpha{_n}$线性表出, 且方式不唯一, 则有无穷多种表出方式.
\end{theorem}

\begin{definition}[线性相关]
    如果$V$是$\mathbb{F}$上的线性空间, $\alpha{_1}, \cdots, \alpha{_n} \in V$, 若\emph{0}被$\alpha{_1}, \cdots, \alpha{_n}$线性表出的方式唯一, 则称$\alpha{_1}, \cdots, \alpha{_n}$线性无关, 否则, 称其线性相关.
\end{definition}

\begin{inference}
    若$\beta$可以由$\alpha{_1}, \cdots, \alpha{_n}$线性表出, 且表出方式唯一$\Longleftrightarrow$ $\alpha{_1}, \cdots, \alpha{_n}$线性无关
\end{inference}

\begin{property}
    \par
    \begin{enumerate}[itemindent=1em]
        \item 若$\alpha{_1}, \cdots, \alpha{_n}$线性无关, 则$\exists 1 < i < n$使得$\alpha{_i}$可以被其余向量线性表出.
        \item 若$\alpha{_1}, \cdots, \alpha{_n}$线性无关, 而$\alpha{_1}, \cdots, \alpha{_n}, \alpha{_{n+1}}$线性相关, 则$\alpha{_{n+1}}$可以被其余向量唯一表出.
        \item 若在$\mathbb{F}^n$上, 向量$\alpha \in \mathbb{F}^n$伸长不改变线性无关性质, 缩短不改变线性相关性质.
    \end{enumerate}
\end{property}

\begin{lemma}[基本引理]
    $V$是$\mathbb{F}$上线性空间, 则若$\alpha{_1}, \cdots, \alpha{_r} \in V$能够被$\beta{_1}, \cdots, \beta{_s}$线性表出, 且$r>s$, 则$\alpha{_1}, \cdots, \alpha{_r}$线性相关.
    同样的可以得到逆否命题.
\end{lemma}

\begin{definition}[极大无关组]
    $V$是$\mathbb{F}$上线性空间, 设$\alpha{_1}, \cdots, \alpha{_n} \in V$, 若其子组设$\alpha{_1}, \cdots, \alpha{_r}$线性无关, 且添加任何其他原组中的向量则变为线性相关, 则称其为原组的一个极大无关组.
\end{definition}

\begin{inference}
    极大无关组包含相同的向量个数
\end{inference}

\begin{definition}[秩]
    极大无关组的个数称为向量组的\emph{秩}
\end{definition}

\begin{example}
    在$C[a, b]$中, $sinx$和$conx$线性无关 
\end{example}

\begin{definition}[Wrongski行列式]
    设$f_1(x), \cdots, f_n(x)$的$n-1$阶导数存在, 则
    \[W(f_1, \cdots, f_n)=\begin{vmatrix}
        f_1(x) & f_2(x) & \cdots & f_n(x) \\
        f_1'(x) & f_2'(x) & \cdots & f_n'(x) \\
        \vdots & \vdots &         & \vdots \\
        f_1^{(n-1)}(x) & f_2^{(n-1)}(x) & \cdots & f_n^{(n-1)}(x)
    \end{vmatrix} \]
    为$f_1, \cdots, f_n$的Wrongski行列式.
\end{definition}

若$W(f_1, \cdots, f_n) \neq 0$, 则$f_1, \cdots, f_n$线性无关.

\begin{example}
    在$\mathbb{F}[x]$中, 若$0 \le i_1 < \cdots < i_n \in \mathbb{Z} $, 则$x^{i_1}, \cdots, x^{i_n} \in \mathbb{F}[x]$线性无关
\end{example}

\begin{example}
    设$V$在$\mathbb{F}$是一个线性空间, 有$\alpha{_1}, \cdots, \alpha{_n} \in V$线性无关, 若有$\beta{_i}=\sum^n_{j=1}a_{ji}\alpha{_j}, 1 \le i \le s$, 即:
    \[(\beta{_1}, \beta{_2}, \cdots, \beta{_s}) = (\alpha{_1}, \alpha{_2}, \cdots, \alpha{_n}) \begin{pmatrix}
        a_{11} & a_{12} & \cdots & a_{1s} \\
        a_{21} & a_{22} & \cdots & a_{2s} \\
        \vdots & \vdots &        & \vdots \\
        a_{n1} & a_{n2} & \cdots & a_{ns}
    \end{pmatrix}\]
    记该矩阵为$A$, 则$rank\{\beta{_1}, \cdots, \beta{_s}\} = rank\{A\}$, 且线性相关性与矩阵$A$对应列的线性相关性相同.
\end{example}

\begin{example}
    在$\mathbb{F}[x]$中, 求$\alpha{_1}=1-x-x^2, \alpha{_2}=x-x^2, \alpha{_3}=-2+x^2$的秩.
\end{example}

\begin{example}
    $\mathbb{C}$在$\mathbb{C}$是线性空间, $i, 1$是线性相关的, 而在$\mathbb{R}$上是线性无关的.
\end{example}

\begin{example}
    定义$\mathbb{Q}[\sqrt[n]{2}]=\{f(\sqrt[n]{2})|f(x)\in \mathbb{Q}[x]\}$, 则$1, \sqrt[n]{2}, \sqrt[n]{2^2}, \cdots, \sqrt[n]{2^{n-1}} \in \mathbb{Q}[\sqrt[n]{2}]$线性无关.
    $f(x)=x^n-1$有唯一的根$\sqrt[n]{2}$, 考虑其与非零多项式$g(x)=\sum^{n-1}_{i=0}a_ix^i$的整除关系.
\end{example}

\begin{definition}[线性无关子集]
    设$V$是$\mathbb{F}$上的线性空间, $S \ne \varnothing \subseteq V$, 如果其任意有限子集都线性无关, 则称$S$是$V$的一个线性无关子集.
\end{definition}

\begin{definition}[极大线性无关子集]
    设$V$是$\mathbb{F}$上的线性空间, $\varnothing \ne S \subseteq V, \varnothing \ne S_1 \subseteq S$, 如果$S_1$是一个线性无关子集, 并且任意$S$中的向量都可以由$S_1$线性表出, 则称$S_1$是$S$的一个极大无关子集.
\end{definition}

\begin{axiom}[基的存在性定理]
    设$V$是$\mathbb{F}$上的线性空间, $\varnothing \ne S \subseteq V, S \ne \{0\}$, 则$S$一定有极大无关子集.
\end{axiom}

\begin{definition}[基]
    $V$的极大无关子集称为$V$的基.
\end{definition}

\begin{example}
    \par
    \begin{enumerate}[itemindent=1em]
        \item $M_{m\times n}(\mathbb{F})$中的基本矩阵.
        \item $C[a, b]$中的基. (无法直接写出)
    \end{enumerate}
\end{example}

\begin{lemma}
    线性空间$V$的任意基所含向量的个数相同.
\end{lemma}

\begin{definition}[线性空间的维数]
    设$S$是$V$的一个基, 则称$S$中向量的个数为维数, 记为$dimV=n$, $V = \varnothing$时, $dimV=0$, 如果$V$的极大无关子集含有向量无穷多, 则$dimV=\infty$.
\end{definition}

\begin{example}
    \par
    \begin{enumerate}[itemindent=1em]
        \item $\mathbb{F}[x]$的维数
        \item 齐次线性方程$AX=0$的解空间的维数
    \end{enumerate}
\end{example}

\begin{inference}
    设$\varnothing \ne V$在$\mathbb{F}$是一个线性空间, 若$dimV=n$, 则其中任意$n$个线性无关的向量都是$V$的一个基.
\end{inference}

\begin{example}
    \par
    \begin{enumerate}[itemindent=1em]
        \item $\mathbb{C}/\mathbb{C}$中, 有$dim\mathbb{C}=1$
        \item $\mathbb{C}/\mathbb{R}$中, 有$dim\mathbb{C}=2$
        \item $\mathbb{R}/\mathbb{Q}$中, 有$dim\mathbb{R}=\infty$
        \item $\mathbb{R}/\mathbb{R}$中, 有$dim\mathbb{R}=1$
    \end{enumerate}
\end{example}

\begin{definition}[坐标]
    设$\alpha{_1}, \cdots, \alpha{_n}$是$n$维线性空间$V$的一个基, 则对$\forall \alpha \in V$, 有
    \[\alpha = \begin{pmatrix}
        \alpha{_1} & \alpha{_2} & \cdots & \alpha{_n}
    \end{pmatrix} \begin{pmatrix}
        x_1 \\
        x_2 \\
        \vdots \\
        x_n
    \end{pmatrix}\]
    称$X$为$\alpha$的在该基下的坐标.
\end{definition}

\begin{example}
    定义建立了一个从$V$到$\mathbb{F}^n$的双射.
\end{example}

\begin{theorem}
    设$V/\mathbb{F}$, 设$\alpha{_1}, \cdots, \alpha{_n}$是$V$的一个基, 设$\beta{_1}, \cdots, \beta{_n} \in V$, 则有:
    \[\begin{pmatrix}
        \beta{_1} & \beta{_2} & \cdots & \beta{_n}
    \end{pmatrix}=\begin{pmatrix}
        \alpha{_1} & \alpha{_2} & \cdots & \alpha{_n}
    \end{pmatrix}A\]
    则$\beta{_1}, \cdots, \beta{_n}$是$V$的一个基$\Longleftrightarrow$有$A$可逆
\end{theorem}

\begin{definition}[转移矩阵]
    由上, 称$A$为从$\alpha{_1}, \cdots, \alpha{_n}$到$\beta{_1}, \cdots, \beta{_n}$的转移矩阵.
    称$GLn(\mathbb{F})=\{\mathbb{F}$上所有可逆的n阶矩阵$\}$.
\end{definition}

\begin{inference}
    在$V/\mathbb{F}$, 由上, 若:
    \[\gamma = \begin{pmatrix}
        \alpha{_1} & \alpha{_2} & \cdots & \alpha{_n}
    \end{pmatrix}X=\begin{pmatrix}
        \beta{_1} & \beta{_2} & \cdots & \beta{_n}
    \end{pmatrix}Y\]
    设$A$是上述的转移矩阵, 则有$Y=A^{-1}X$.
\end{inference}

\section{ 线性映射和同构 }

\begin{definition}[线性映射]
    设$V_1, V_2/\mathbb{F}$, 设映射$f:V_1 \to V_2$, 满足:
    \begin{enumerate}[itemindent=1em]
        \item $f(\alpha + \beta) = f(\alpha) + f(\beta), \forall \alpha, \beta \in V_1$
        \item $f(k\alpha) = kf(\alpha), , \forall \alpha \in V_1, \forall k \in \mathbb{F}$
    \end{enumerate}
    则称$f$为从$V_1$到$V_2$的线性映射. 记所有此线性映射的集合为$Hom(V_1, V_2)$或$L(V_1, V_2)$
\end{definition}

\begin{example}
    \par
    \begin{enumerate}[itemindent=1em]
        \item 设$V_1, V_1/\mathbb{F}, f: \alpha \mapsto 0, \alpha \in V_1$, 即$0 \in Hom(V_1, V_2)$. 这个例子说明$Hom(V_1, V_2)$不会是一个$\varnothing$.
        \item 将$V, dimV=n$中的向量映射到$\mathbb{F}^n$中的该向量的坐标的映射, 是一个线性映射.
        \item 在$M_{m\times n}(\mathbb{F})$中, 将一个矩阵映射到其的转置.
    \end{enumerate}
\end{example}

\begin{definition}[线性变换]
    在一个线性映射中, 若$V_1 = V_2$, 则称该$f$是一个线性变换.
\end{definition}

\begin{definition}[线性映射的矩阵]
    设$V_1, V_2, \ dimV_1=n, \ dimV_2=m$是两个有限维线性空间, 设$\alpha{_1}, \cdots, \alpha{_n}$以及$\beta{_1}, \cdots, \beta{_m}$分别是两个线性空间的基, 设有线性映射$f$, 有:
    \[\begin{pmatrix}
        f(\alpha{_1}) & f(\alpha{_2}) & \cdots & f(\alpha{_n})
    \end{pmatrix}=\begin{pmatrix}
        \beta{_1} & \beta{_2} & \cdots & \beta{_m}
    \end{pmatrix}A, \ A \in M_{m\times n}(\mathbb{F})\]
\end{definition}

\begin{example}
    设$V_1, V_2\ / \mathbb{F}$, 取定$\alpha{_1}, \cdots, \alpha{_n}$以及$\beta{_1}, \cdots, \beta{_m}$分别是两个线性空间的基, 则有单射$\sigma: Hom(V_1, V_2) \to M_{m\times n}(\mathbb{F})$.
\end{example}

\begin{theorem}[线性映射存在性与唯一性定理]
    设$V_1, V_2\ / \mathbb{F}, \ \ dimV_1 = n$, 则取定$V_1$的一个基$\alpha{_1}, \cdots, \alpha{_n}$, 取定$V_2$中的一个向量组$\beta{_1}, \cdots, \beta{_n}$, 则存在唯一的$f \in Hom(V_1, V_2)$, 使得$f(\alpha{_i})=\beta{_i},\ 1 < i < n$.
\end{theorem}

\begin{example}
    $f \in Hom(V_1, V_2)$, 关于两个线性空间的两个基(有限)的矩阵为$A$, 设$\alpha \in V_1$, 其坐标为$X$, 则有其在$V_2$中的坐标为$AX$
\end{example}

\begin{example}
    由上例, 设$im(f)=\{f(\alpha) \in V_2|\alpha \in V_1\}$, 则$\beta \in im(f)$ $\Longleftrightarrow$ $\beta$的坐标$Y$可以由$A$的列向量线性表出
\end{example}

\begin{example}
    有$kerf=\{\alpha \in V_1|f(\alpha)=0\}=\{\alpha \in V_1|AX=0\}$, 可见$kerf$不是空集, 一定有$0$.
\end{example}

\begin{inference}
    $f \in Hom(V_1, V_2)$, 则:
    $f$是满射$\Longleftrightarrow$$im(f)$中向量可以由$A$的列向量线性表出$\Longleftrightarrow$$r(A)$等于$dimV_2$.
    $f$是单射$\Longleftrightarrow$$kerf=\{0\}$.
    两者都是有限维线性空间, 则$f$单且满$\Longleftrightarrow$$A$满秩.
\end{inference}

\begin{example}
    设$f \in Hom(V_1, V_2)$, 其中$\alpha{_1}, \cdots, \alpha{_n}$和$\alpha'{_1}, \cdots, \alpha'{_n}$是$V_1$的基, $\beta{_1}, \cdots, \beta{_m}$和$\beta'{_1}, \cdots, \beta'{_m}$是$V_2$的基, 有:
    \[\begin{pmatrix}
        f(\alpha{_1}) & f(\alpha{_2}) & \cdots & f(\alpha{_n})
    \end{pmatrix}=\begin{pmatrix}
        \beta{_1} & \beta{_2} & \cdots & \beta{_m}
    \end{pmatrix}A, \ A \in M_{m \times n}(\mathbb{F})\]
    \[\begin{pmatrix}
        f(\alpha'{_1}) & f(\alpha'{_2}) & \cdots & f(\alpha'{_n})
    \end{pmatrix}=\begin{pmatrix}
        \beta'{_1} & \beta'{_2} & \cdots & \beta'{_m}
    \end{pmatrix}B, \ B \in M_{m \times n}(\mathbb{F})\]
    \[\begin{pmatrix}
        \alpha'{_1} & \alpha'{_2} & \cdots & \alpha'{_n}
    \end{pmatrix}=\begin{pmatrix}
        \alpha{_1} & \alpha{_2} & \cdots & \alpha{_n}
    \end{pmatrix}P, \ P \in M_n(\mathbb{F})\]
    \[\begin{pmatrix}
        \beta'{_1} & \beta'{_2} & \cdots & \beta'{_n}
    \end{pmatrix}=\begin{pmatrix}
        \beta{_1} & \beta{_2} & \cdots & \beta{_n}
    \end{pmatrix}Q, \ Q \in M_m(\mathbb{F})\]
    有$B=Q^{-1}AP$. $A$和$B$是在同一个线性映射不同基下的矩阵.
\end{example}

\begin{definition}[同构映射]
    设$V_1, V_2 / \mathbb{F}$, 设其线性映射$f$单且满, 则称其为同构映射.
\end{definition}

\begin{inference}
    $f$为同构映射$\Longleftrightarrow$$f$的矩阵$A$可逆.
\end{inference}

\begin{lemma}
    设$f \in Hom(V_1, V_2)$, 则有$f$是同构映射$\Longleftrightarrow$$\exists \ g \in Hom(V_1, V_2) \rightarrow g \cdot f = id_{V_1}, f \cdot g = id_{V_2}$    
\end{lemma}

\begin{example}
    \begin{enumerate}[itemindent=1em]
        \item $M_{m \times n}(\mathbb{F}) \cong M_{n \times m}(\mathbb{F})$
        \item $D: \ \mathbb{F}[x] \to \mathbb{F}[x], \ f(x) \mapsto f'(x)$, 有$KerD = \mathbb{F}$, 其是满射但是不是单射.
    \end{enumerate}
\end{example}

\section{ 子空间 }

\begin{definition}[子空间]
    设线性空间$V$的一个子集$W$关于$V$的运算封闭, 则称$W$是$V$的线性子空间. 特别地, 任意线性空间的子空间含零空间.
\end{definition}

\begin{example}
    \begin{enumerate}[itemindent=1em]
        \item $M_{n}(\mathbb{F})$中的对称矩阵组成的集合.
        \item $M_n(\mathbb{F})$中秩为1的矩阵组成的集合.
        \item $\mathbb{F}[x]$中小于特定阶数的多项式集合.
        \item $\mathbb{F}[x]$中同构于其子空间$W$: $x^i \mapsto x^{2i},\ i \in \mathbb{N}$.
        \item $\mathbb{R^+} \subset \mathbb{R}$关于乘法和幂.
    \end{enumerate}
\end{example}

\begin{property}
    由上, $dimW \le dimV$, 特别地, 有$dimW = dimV \leftrightarrow W = V$.
\end{property}

\begin{theorem}[基的扩充定理]
    $W$是$V$的子空间, 则$W$的任意一个基可以扩充为$V$的一个基.
\end{theorem}

\begin{definition}[生成子空间]
    设$V/\mathbb{F}$, $\alpha{_1}, \ \cdots, \ \alpha{_s} \in V$, 称$<\alpha{_1}, \cdots, \alpha{_s}>=\{\beta \in V|\beta=\sum^s_{i=1}k_i\alpha{_i}, \ k_i \in \mathbb{F}\}$为$\alpha{_1}, \cdots, \alpha{_s}$张成的子空间.
\end{definition}

\begin{example}
    求$dim<1, \ x+1, \ x^2+x+1, \ x^2+x-2>$
\end{example}

\begin{example}
    设$V_1,V_2/\mathbb{F}, \ f \in Hom(V_1, V_2)$, 则$kerf$是$V_1$的子空间, $im(f)$是$V_2$的子空间.
\end{example}

\begin{definition}[同构]
    若两个线性空间之间存在同构映射, 则称两个线性空间同构(等价关系), 记为$V_1 \cong V_2$.
\end{definition}

\begin{theorem}[同构定理]
    两个都是有限维的线性空间同构$\longleftrightarrow$两个线性空间的维数相等(设为n), 特别地, 其同构于$\mathbb{F}^n$
\end{theorem}

\begin{example}
    设$dimV_1 = n,\ dimV_2 = m$则有$Hom(V_1, V_2)\cong M_{m\times n}(\mathbb{F})$, 特别地, $dimM_{m\times n}(\mathbb{F})=dimHom(V_1, V_2)=mn$
\end{example}

\section{ 子空间的运算 }

\begin{theorem}[子空间的维数公式]
    设$V_1, V_2/\mathbb{F}, \ f\in Hom(V_1, V_2)$, 其中$dimV_1=n$$\Longrightarrow$$dimkerf + dimim(f) = n$
\end{theorem}

\begin{inference}
    由上, 若$dimV_1=dimV_2=n$, 则有$f$是单射$\Longleftrightarrow$$f$是双射$\Longleftrightarrow$$f$是同构映射.
\end{inference}

\begin{statement}[子空间的交]
    设$V_1, V_2$都是$V$的子空间, 子空间的交为$V_1\cap V_2$.
    子空间的交仍是子空间, 多个子空间的交仍是子空间.
\end{statement}

\begin{example}
    \begin{enumerate}[itemindent=1em]
        \item 如何求两个子空间的交及其维数?
        \item 若$dimV=n$, 则$V$可以写成若干$n-1$维子空间的交.
        \item (线性映射保子空间), $W$是$V_1$的子空间, 则$f(W)$是$V_2$的子空间.
    \end{enumerate}
\end{example}

\begin{lemma}[子空间的并]
    两个线性子空间$V_1, V_2$的并是子空间$\Longleftrightarrow$$V_1\subseteq V_2$或$V_2\subseteq V_1$.
    可以看出线性子空间的并意义不大.
\end{lemma}

\begin{theorem}
    线性空间$V$无法等于其有限多个线性真子空间的并.
\end{theorem}

\begin{statement}
    线性子空间的补$W^c$一定不是子空间.
\end{statement}

\begin{lemma}
    线性子空间的和$V_1+V_2\triangleq{\{\alpha{_1}+\alpha{_2}|\alpha{_1}\in V_1, \ \alpha{_2}\in V_2\}}$是子空间.
\end{lemma}

\begin{example}
    \[V=V_1+V_2+\cdots+V_s\Longleftrightarrow \forall \ \alpha \in V, \exists \ \alpha{_1}\in V_1, \cdots, \alpha{_s}\in V_s\to\alpha=\sum^s_{i=1}\alpha{_i}\].
\end{example}

\begin{example}
    $V=M_n(\mathbb{F}), \ V_1=\{A\in V|A=A'\}, \ V_2=\{A\in V|A=-A'\}$, 有$V=V_1+V_2$, 由于$A=\frac{A+A'}{2}+\frac{A-A'}{2}$.
\end{example}

\begin{example}
    \[ V_1=\{\alpha \in V|f(\alpha)=\alpha\},\ V_2=\{\alpha \in V|f(\alpha)=-\alpha\},\ f^2=id_V\Longrightarrow V=V_1+V_2 \].
\end{example}

\begin{statement}
    \[<\alpha{_1}, \cdots, \alpha{_s}>+<\beta{_1}, \cdots, \beta{_t}>=<\alpha{_1}, \cdots, \alpha{_s}, \beta{_1}, \cdots, \beta{_t}>\]
    注意子空间的加不具有消去律.
\end{statement}

\begin{theorem}[子空间的维数公式]
    设$V_1, V_2$是$V$的子空间, 且其维数都有限, 则$dim(V_1+V_2)=dimV_1+dimV_2-dim(V_1\cap V_2)$
\end{theorem}

\section{ 直和 }

\begin{definition}[直和]
    若$\forall \ \alpha \in \sum^s_{i=1}V_i, \ \exists \ \alpha{_i}\in V_i\to \alpha = \sum^s_{i=1}\alpha{_i}$, 且表示方法唯一, 则称$\sum^s_{_i=1}V_i$是一个直和.
\end{definition}

\begin{inference}
    $\sum^s_{i=1}V_i$是直和$\Longleftrightarrow$零的写法唯一.
\end{inference}

\begin{statement}
    $V_1+V_2$是直和$\Longleftrightarrow V_1\cap V_2=\{0\}$
\end{statement}

\begin{example}
    设$V_1=\{f(x)\in C[-\infty, +\infty]|f(a)=0\}, \ V_2=\{f(x)\in C[-\infty, +\infty]|f(b)=0\}$, 则$V_1+v_2$是直和?
\end{example}

\begin{theorem}
    设$V_i, \ 1\le i \le s$的一个基为$\alpha{_{i1}}, \cdots, \alpha{_{it_i}}$, 则$\sum^s_{i=1}V_i$是直和$\Longleftrightarrow\ dim\sum^s_{i=1}V_i=\sum^s_{i=1}dimV_i$
\end{theorem}

\begin{example}
    若有$V_1\oplus V_2=V_1 \oplus V_3\Longrightarrow dimV_2=dimV_3\Longrightarrow V_2\cong V_3$.
    注意不能够推导出$V_2=V_3$.
\end{example}

\begin{example}
    $f\in Hom(V,V)$则有$kerf+im(f)$是直和.
\end{example}

\begin{theorem}
    设$V_1$是$V$的子空间, 则一定存在$V_2$是$V$的子空间, 使得$V=V_1\oplus V_2$.
\end{theorem}

\begin{definition}[补空间]
    称上述定理中的$V_2$是$V_1$在$V$中的补空间.
\end{definition}

\begin{definition}[嵌入]
    设$V_1$是$V$的子空间, 则有映射$\tau: V_1 \to V, \ \alpha \mapsto \alpha$, 称为从$V_1$到$V$的嵌入.
\end{definition}

\begin{definition}[线性投影]
    取$V_1$在$V$中的补空间$V_2$, 有$\forall \ \alpha \in V, \ \alpha=\alpha{_1}+\alpha{_2}, \ \alpha{_1}\in V_1, \alpha{_2}\in V_2$, 表示唯一, 称映射$p_{V_1}: V\to V_1, \ \alpha \mapsto \alpha{_1}$为$V$沿$V_1$的线性的投影.
\end{definition}

\begin{theorem}[准素分解定理]
    设$V/\mathbb{F}$, 若有$\rho \in EndV, \ f(x)\in\mathbb{F}[x], \ f(\rho)=0, \ f(x)=f_1(x)\cdots f_s(x)$且$f_i(x)$两两互素, 则有:
    \[V=kerf_1(\rho)\oplus\cdots\oplus kerf_s(\rho)\]
\end{theorem}

\begin{example}
    $V/\mathbb{F}$, 设$f\in Hom(V,V)$, 则$V=ker(f-3id_V)\oplus ker(f+2id_V)\Longleftrightarrow f^2-f-6id_V=0$
\end{example}

\begin{example}
    线性变换保空间分解: $V=V_1\oplus V_2, \ dimV=n, \ f\in Hom(V, V),\ V=f(V_1)\oplus f(V_2)\Longrightarrow f$是同构映射.
\end{example}

\begin{example}
    设$V,W/\mathbb{F}$, 且$V=V_1\oplus\cdots\oplus V_s, \ W=W_1\oplus\cdots\oplus W_t$, 若有:
    \[V_i\stackrel{l_i}{\to}V\stackrel{f}{\to}W\stackrel{p_{W_j}}{\to}W_j\],
    记$f_{ji}\triangleq p_{W_j}\cdot f \cdot l_i$, 若设线性空间为有限维, 则可以得到分块矩阵的具体意义, 略.
\end{example}
