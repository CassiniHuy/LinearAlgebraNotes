%\usepackage{minitoc}

% 使用 geometry 宏包定制页面版式
% pdf 电子文档版式
\usepackage[a4paper,left=3.5cm,right=3.5cm, bottom=3.5cm,top=3.5cm]{geometry}
% 印刷版式,左右页的内外边距不同,以抵消装订线对页边空白的影响。
%\usepackage[a4paper,inner=4cm,outer=2.5cm, bottom=3.5cm,top=3.5cm]{geometry}

\usepackage{enumitem}
% \usepackage{emptypage}
% 定制目录样式的宏包
\usepackage{etoc}

% 定制日期时间格式的宏包
\usepackage[yyyymmdd]{datetime}
\renewcommand{\dateseparator}{-}

% 数学必备宏包
\usepackage{amsmath}
\usepackage{amssymb}

% 定理和证明环境
\usepackage{amsthm}

\usepackage{makecell}

% 插图宏包,提供插图命令 \includegraphics
\usepackage{graphicx}

% 好看的向量箭头符号,命令是 \vv
\usepackage{esvect}

% 数学花体,命令是\mathscr
\usepackage{mathrsfs}

% 数学粗体,用于向量或矩阵等,命令是\bm
\usepackage{bm}

% 改善表格排版质量的宏包
\usepackage{booktabs}

% 使目录和各种引用具有超链接效果
\usepackage[colorlinks,linkcolor=black,CJKbookmarks=true,bookmarksnumbered]{hyperref}

% 定制插图和表格的标题的宏包
\usepackage[font=small,labelfont=bf,labelsep=space]{caption}

